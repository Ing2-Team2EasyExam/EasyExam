\begin{authorship}
T. Biedl
\end{authorship}
\begin{usage}
Fall 2005, Assignment 7
\end{usage}

We say that a graph $G=(V,E)$ has a {\em multiple edge} if there
exist two distinct vertices $v,w$ in $G$ and at least two edges
in $G$ that have $v$ and $w$ as their endpoints.

Assume that $G$ is stored with adjacency lists.  Show how to test
whether $G$ contains a multiple edge in $O(m+n)$ time.

\begin{hint}
We have $n$ vertices, and you may assume that the vertices are
already numbered as $1,\dots,n$.  How quickly can you hence
eliminate duplicates from a list of vertices?
\end{hint}

\begin{solution}
The idea is to store all edges in a list and apply radix sort to
the list.  Then a multiple edge will show up twice in the list, which,
since it is sorted, we can test by scanning the list.  We can find all
edges to put in the list by scanning the adjacency list of each vertex,
but to avoid putting edges in repeatedly, we will only add it to the
list of edges if the first endpoints has a smaller number than the second.
Here is the
pseudo-code.

\begin{verbatim}
Assume the vertices are numbered 1...n
Initialize an empty list L
For each vertex i {
    Let Adj(i) be the adjacency list of vertex i
    For each j in Adj(i)
        if i < j
            add (i,j) to L
}
Sort L in lexicographic order using radix sort
Let (i,j) be the first element in L
while not at the end of L  {
    Let (k,l) be the next element in L
    if i==k and j==l, then {i,j} is a multiple edge, stop
    (i,j) <- (k,l)
}
\end{verbatim}

For the correctness, note that we have a multiple edge if and only
if for some vertex $i$, we have some index $j>i$ that appears
twice in the adjacency list of $i$.  So $(i,j)$ will appear twice in
$L$. After sorting $L$ in lexicographic order, therefore $(i,j)$ appears
twice at consecutive positions in $L$, and our algorithm will find it.
On the other hand, if $(i,j)$ appears twice in $L$, then $j$ must have
been twice in the adjacency list of $i$, so this was a multiple edge.

For the time complexity, we spend $O(n)$ time to access all vertices,
and $O(deg(i))$ time for each vertex $i$ in the for-loop, so this is $O(m+n)$
total.  There are $m$ elements in $L$, and each of them is a pair $(i,j)$
with $1\leq i,j\leq n$.  So sorting $L$ by one digit takes $O(m+n)$ time,
and there are two digits, so sorting $L$ takes $O(m+n)$ time.  Finally,
scanning through $L$ takes $O(m)$ time, so the total runtime is $O(m+n)$.
\end{solution}

\begin{markingScheme}
15 marks for the algorithm, 5 marks for why it works (doesn't have to be
as precise as what I write), 5 marks for analysing the time complexity.
Note that one {\em could} put all edges twice into the edge-list and it
would still work (as long as they're careful to keep the order of endpoints
intact.)

For slower algorithms, reduce as follows:
\begin{itemize}
\item 20 marks (10+5+5) if they apply CountSort or similar on each Adjacency
	list.
	(This is not correct, since it actually takes $O(\deg(v)+n)$ per vertex,
	but I'd expect this to be done quite frequently.)
\item 15 marks (5+5+5) for anything that tries to use CountSort in a clever way
	to avoid regular sorting/elimination of duplicates.
\item 10 marks (5+3+2) for any solution in $O(n+m\log m)$ time
	(e.g., compute all edges, then sort and eliminate.)
\item 5 marks (3+1+1) for anything that works in $o(m^2+n)$ time and
	doesn't fall in the above categories.
\end{itemize}
\end{markingScheme}
